\section{Mathematical Modelling}
\label{mathModel}

\subsection{Numerical Details}
\paragraph{Mesh} The domain is a cylinder with a length of $35D_{jet}$ and diameter of $28D_{jet}$. The LES mesh is discretised into $2.24$M cells approximately ($77(r)x64(\theta)x480(z)$.
The secondary mesh, used to control the particle resolution, has around $19.2$K cells.

\paragraph{Particle Resolution} Maximum of $0.43$M particles exist in  the computational domain with $N_{pc}=20$ and resolution $1L/6E$.

\paragraph{Chemical Mechanism} A $30$-species skeletal mechanism, based on GRI 3.0, developed by Lu and Law is used \cite{Lu2008}.

\paragraph{Inlet B.C.} The \textbf{jet} was imposed with an instantaneous velocity profile from precursor pipe flow simulation with the \textbf{mean and rms scaled} up with the centreline experimental measurements. \\
The \textbf{pilot} was provided top-hat velocity profile using measured bulk velocity with imposed artificial turbulent fluctuations using the Turbulent Spot method \cite{Kroger2018}. The \textbf{co-flow} was laminar with experimental bulk velocities. All other scalars were provided uniform boundary values from the measurements.

\paragraph{Integration Schemes} A second order central differencing scheme is used for the convective terms in the finite volume momentum equation while the scalar equations use a second order accurate NVD schemes \cite{Jasak1996}. The finite volume equations are marched in time using fully implicit Crank-Nicolson scheme. The particle ODE's, on the other hand, are limited to first order Euler integration.

\paragraph{Mixing Time Scale} An anisotropic minor mixing time scale \cite{Vo2017a} is used for the all the simulations. The three variations of the mixing time scale are shown in Table~\vref\ref{tab:MixTime}. The mixing pairs were advanced with a harmonic mean of the time scales of the two particles.

Table \ref{tab:MixTime}
